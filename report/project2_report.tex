\documentclass[12pt]{article}
\usepackage[swedish,english]{babel}
\usepackage[utf8x]{inputenc}
\usepackage{amsmath}
\usepackage{graphicx}
\usepackage{float} %for insert graph
\usepackage{lscape}
\usepackage{rotating}
\usepackage[colorinlistoftodos]{todonotes}
%\usepackage[margin=1in]{geometry} %set page margin
\usepackage[bottom=1.25in, top=1.25in]{geometry}
%\addtolength{\topmargin}{0.25in}
%\addtolength{\bottommargin}{0.25in}
\usepackage{hyperref} %insert to link to email address
\usepackage{setspace}
\setlength\parindent{24pt} %set indentation
\usepackage{amssymb} %for maths symbols
\usepackage{cases} %for numbering in cases
  %define matlab style
\usepackage{listings}
\usepackage{color} %red, green, blue, yellow, cyan, magenta, black, white
\definecolor{mygreen}{RGB}{28,172,0} % color values Red, Green, Blue
\definecolor{mylilas}{RGB}{170,55,241}
\usepackage{grffile} %to avoid showing the file names of figures

\lstset{language=Matlab,%
    %basicstyle=\color{red},
    breaklines=true,%
    morekeywords={matlab2tikz},
    keywordstyle=\color{blue},%
    morekeywords=[2]{1}, keywordstyle=[2]{\color{black}},
    identifierstyle=\color{black},%
    stringstyle=\color{mylilas},
    commentstyle=\color{mygreen},%
    showstringspaces=false,%without this there will be a symbol in the places where there is a space
    numbers=left,%
    numberstyle={\tiny \color{black}}% size of the numbers
    %numbersep=9pt, % this defines how far the numbers are from the text
    %emph=[1]{for,end,break},emphstyle=[1]\color{red}, %some words to emphasise
    %emph=[2]{word1,word2}, emphstyle=[2]{style},    
}


\begin{document}

%%%%%%%%%%%
%%TITLE PAGE%%%
%%%%%%%%%%%%
\begin{titlepage}

\newcommand{\HRule}{\rule{\linewidth}{0.5mm}} % Defines a new command for the horizontal lines, change thickness here

\center % Center everything on the page
 
%----------------------------------------------------------------------------------------
%   HEADING SECTIONS
%----------------------------------------------------------------------------------------
%Logo First
%----------------------------------------------------------------------------------------
%   LOGO SECTION
%----------------------------------------------------------------------------------------

\includegraphics{UU_logo.eps}\\[0.5cm] % Include a department/university logo - this will require the graphicx package
 
\textsc{\huge Modelling complex systems}\\[0.5cm] % Major heading such as course name


%----------------------------------------------------------------------------------------
%   TITLE SECTION
%----------------------------------------------------------------------------------------

\HRule \\[0.4cm]
{\huge \bfseries Project 2}\\[0.2cm] % Title of your document
{\Large Population Dynamics \par Groups of Friends \par Network Epidemics \par Flocks and Predators }
\\

\HRule \\[0.4cm]
%{\Large Population Dynamics \par Groups of Friends \par Network Epidemics \par Flocks and Predators }

 
%----------------------------------------------------------------------------------------
%   AUTHOR SECTION
%----------------------------------------------------------------------------------------

{\huge Peili Guo\\} %insert your name 
{\large \href{mailto:Peili.Guo.7645@student.uu.se}{Peili.Guo.7645@student.uu.se}}
\\[2cm] %insert page break length

%----------------------------------------------------------------------------------------
%   DATE SECTION
%----------------------------------------------------------------------------------------

{\Large \today}\\[2cm] % Date, change the \today to a set date if you want to be precise


%----------------------------------------------------------------------------------------

\vfill % Fill the rest of the page with whitespace

\end{titlepage}

\newpage
%%%%%%%%%%%%
%%%start report 
%%%1. population dynamics
%%%%%%%%%%%

\section{Population Dynamics}
\doublespacing
In this part, we model the population with a stochastic model. There are n resource sites in the model world, and at time t =0, the population is $A_{0}$ and they are assigned randomly to one resource site. At each t step, the population rules are, if there are exactly two individuals on the same site. They reproduce b offsprings and these offsprings are assigned randomly to resources sites. If the number of individuals on a resources sites is any number other than 2, no offspring will be reproduced. 

\subsection{Matlab model}

To begin, we can run this with different parameters and simulate in Matlab and observe the total number of population at different time step. The different parameters are: 

\begin{enumerate}
\item b: number of offspring if reproduce
\item n: total of number of resource sites
\item $A_{0}$: initial population
\item t: the time steps we want to simulate the model
\end{enumerate}


When set the initial population to a small number relative to the resource site, it would be difficult to have 2 individuals at the same site for reproduce, and even if they reproduce a large number of offspring the population dies out very quick. 
below are some plots showing the total number of population at different time steps with different parameters. 

\begin{figure}[H] %figure 1 at x0=10 b= 50
\centering
\includegraphics[width = 12 cm, height = 9cm]{single_run n1000 x010 b50.png}
\caption{initial population of 10 b = 50 not reproducing}
\label{fig:p1s1}
\end{figure}

\begin{figure}[H] %figure 2 x0 = 50 b =30 
\centering
\includegraphics[width = 12 cm, height = 9cm]{single_run n1000 x050 b30.png}
\caption{initial population of 50 and b =30 not reproducing}
\label{fig:p1s2}
\end{figure}

\begin{figure}[H] %figure3 x0 = 50 b= 50
\centering
\includegraphics[width = 12 cm, height = 9cm]{single_run n1000 x050 b50.png}
\caption{initial population of 50 and b = 50, in the beginning reproduce but quickly dies}
\label{fig:p1s3}
\end{figure}

\begin{figure}[H] %figure4 x0 = 100 b = 20
\centering
\includegraphics[width = 12 cm, height = 9cm]{single_run n1000 x0100 b20.png}
\caption{initial population of 100 and b = 20, it dies}
\label{fig:p1s4}
\end{figure}

\begin{figure}[H] %figure5 x0 = 100 b = 50
\centering
\includegraphics[width = 12 cm, height = 9cm]{single_run n1000 x0100 b50.png}
\caption{initial population of 100 and b = 50, it reproduce in the beginning but quickly dies}
\label{fig:p1s5}
\end{figure}

\begin{figure}[H] %figure 6 x0 1000 b = 10
\centering
\includegraphics[width = 12 cm, height = 9cm]{single_run n1000 x01000 b10.png}
\caption{initial population of 1000 and b = 10, it show that the population oscillates around 2500}
\label{fig:p1s6}
\end{figure}

\begin{figure}[H] %figure 7 x0 1000 b =20
\centering
\includegraphics[width = 12 cm, height = 9cm]{single_run n1000 x01000 b20.png}
\caption{initial population of 1000 and b = 20, the populations starts to get chaotic}
\label{fig:p1s7}
\end{figure}

\begin{figure}[H] %figure 8 x0 1000 b =50
\centering
\includegraphics[width = 12 cm, height = 9cm]{single_run n1000 x01000 b50.png}
\caption{initial population of 1000 and b = 500, the populations dies very quick}
\label{fig:p1s8}
\end{figure}


To further study this model, we run it with $A_{0} = 1000$, n= 1000, b = 1, 2, 3, 4, 5,...,48, 49, 50. and at each b value, run the simulation 100 times and plot a phase transition diagram. and we can see that when b = 5-15, the population increases steady and oscillates around a number. When $15<b<35$, the population will get more chaotic and at one time a lot of sites are reproducing, and then the next time steps, they dies because of overcrowding. When $b>35$ the population will dies very quick due to overcrowding. 


\begin{figure}[H] %figure 9 phase transition diagram
\centering
\includegraphics[width = 16 cm, height = 13cm]{q1_phase_r100.png}
\caption{initial population of 1000 and b = 500, the populations dies very quick}
\label{fig:p1q1pd}
\end{figure}

\newpage
\subsection{Mean field model}
We assume the sites are independent and number of individuals at sites is poisson distributed. 
I plot selected mean-field model here for b = 5, 10, 15, 20, 35, 50. The model are after the plot

\begin{figure}[H] %figure 10 mean field model
\centering
\includegraphics[width = 18 cm, height = 14cm]{meanfield_model.png}
\caption{mean field model with selected b value}
\label{fig:p1q1mf}
\end{figure}

\begin{numcases}{}
	E(A_{t+1}|A_{t}) = n \times \sum_{k=0}^{n} P_{k} \times \phi_{k} \\
	\phi_{k} = \begin{cases}
		b, if k =2\\
		0, if k \neq 2
	\end{cases}
\end{numcases}



following the mean field model, and the population only reproduce if there are exactly 2 individuals at same site, we can write the following:

\begin{equation}
	A_{t+1} = n \times P(2 at site, A_{t}) \times b\\
\end{equation}

\begin{equation}
	A_{t+1} = n \times \frac{(\frac{A_{t}}{n})^{2} \times e^{-\frac{A_{t}}{n}}} {2} \times b
\end{equation}

For steady state, we have $A_{t+1} = A_{t}$.\\

\begin{equation}
	A_{t} = n \times \frac{(\frac{A_{t}}{n})^{2} \times e^{-\frac{A_{t}}{n}}} {2} \times b
\end{equation}

When $A_{t} = 0$, left hand side always equal to right hand side, they are both 0. When $A_{t}  \neq 0$, we have

\begin{equation} \label{eq1}
	1 = \frac{b \times A_{t}}{2n} \times e^{-\frac{A_{t}}{n}}
\end{equation}

Let's rearrange equation (6) and let $\frac{A_{t}}{n} = x $, we have:

\begin{equation}
	1 = \frac{b}{2} \times x \times e^{-x}
\end{equation}

\begin{equation}
	x \times e^{-x} = \frac{b}{2}
\end{equation}

To find the conditions in terms of b for the existence of two further non-zero steady states, we need to find that $f1(x) = x \times e^{-x}$  has intersection with a horizontal line. b need to fulfill the condition that $\frac{2}{b} < 0.3679$, therefore we get: $b > 5.4366$, as b needs to be integer, we get $b > 5$. 


\begin{figure}[H] %figure 11 plot of function
\centering
\includegraphics[width = 16 cm, height = 13cm]{q1_2plotfunction.png}
\caption{plot of f1(x) = x * exp(-x)}
\label{fig:p1q1plotfun}
\end{figure}


\newpage
\subsection{Lyapunov exponent}

From the previous section, we have derived the equation for $A_{t}$. that we have 
\begin{equation}
	f(A_t) = n \times \frac{(\frac{A_{t}}{n})^{2} \times e^{-\frac{A_{t}}{n}}} {2} \times b
\end{equation}

\begin{equation}
	f(a) = \frac{b}{2n} \times a^{2} \times e^{-\frac{a}{n}}
\end{equation}

\begin{equation}
	f'(a) = \frac{b}{2n} \times (2a \times e^{-\frac{a}{n}} + a^{2} \times e^{-\frac{a}{n}} \times(- \frac{1}{n}))
\end{equation}

\begin{equation}
	f'(a) = \frac{b}{2n} \times (2a \times e^{-\frac{a}{n}} - \frac{a^{2}}{n} \times e^{-\frac{a}{n}} )
\end{equation}

\begin{equation}
	|\Delta a_{n}| = |\Delta a_{0}| e^{\lambda n}
\end{equation}

to compute Lyapunov exponent $\lambda$ numerically, we use the following:
\begin{equation}
	\lambda = \frac{1}{n} \sum_{t=0}^{n-1} ln |f'(a_{t})|
\end{equation}

The Lyapunov exponent are plotted below: we can see that when $\lambda > 0$ the diverges, that's when b is between 21-30, the population become chaotic. When $b \leq 5$ and $b \geq 31$ $\lambda$ is -infinity that means the system converge very fast, and this explains why the population become extinct very fast for large b. 

\begin{figure}[H] %figure 12 lyaponov exponent
\centering
\includegraphics[width = 16 cm, height = 13cm]{lyapunov_exp.png}
\caption{plot of Lyapunov exponent}
\label{fig:p1q1lya}
\end{figure}


%%%%%%%%%%%%%%%%%%%%%%%%%%part 2
%%%%%%%groups of friends

\newpage
\section{Groups of friends}
\doublespacing
In this part, a model is used to simulate the spread of internet memes. There are 3 different states, resting(0), sharing(1) and bored(2). The rules for the next time step are:

\begin{enumerate}
\item with probability p = 0.001, a person at rest will discover a new meme and become a sharer.(0 $\rightarrow$ 1 with p = 0.001) 
\item with probability q = 0.01, a person sharing(1) will pick one person completely at random from the population to share the memes with. if the random person is at rest(0), that person will become a sharer(1), if that person is bored(2), then the sharing person will become bored(2). 
\item bored(2) stays bored(2) forever. (2 is always 2).
\end{enumerate}

\subsection{some simulations in matlab}
The simluation in matlab will run the model 1000 times with a population of 1000 to time at 2000 and show the change of number of resting, sharing and bored person over time. The initial condition is that there are one person sharing and one bored person. and below are the graphs showing the simulation. 

\begin{figure}[H] %1000 meme run >.<
\centering
\includegraphics[width = 12 cm, height = 9cm]{memes_sim_1000times.png}
\caption{simulation of spread of memes showing number of bored, sharing, resting person over time}
\label{fig:mem}
\end{figure}

The mean field difference equation model for the sharing of meme is: \par
Bored(B), Sharing(S), Resting(R), population(N).\par
\begin{numcases}{ }
	B(t+1) = B(t) + S(t)*q*B(t)/N\\
	S(t+1) = S(t) + p*R(t) - S(t)*q*B(t)/N + S(t)*q*R(t)/N\\
	R(t+1) = R(t) - R(t)*p - S(t)*q*R(t)/N
\end{numcases}

The figure below shows both the simulation and the mean field model
\begin{figure}[H] %1000 meme run >.<
\centering
\includegraphics[width = 16 cm, height = 13cm]{memes_withmodel1000.png}
\caption{simulation of spread of memes showing number of bored, sharing, resting person over time with mean field model}
\label{fig:meme_sim_model}
\end{figure}

\begin{figure}[H] %phase transition
\centering
\includegraphics[width = 16 cm, height = 13cm]{phasetransition.png}
\caption{phase transition of total sharing person with simulation with t = 1000 and different B(0) condition}
\label{fig:meme_phase_transition_1}
\end{figure}

\begin{figure}[H] %result plot of individual
\centering
\includegraphics[width = 16 cm, height = 13cm]{indi_phase.png}
\caption{plot of sharing person in individual runs at t =1000 for 100 simulations of B0 = 1:1:999}
\label{fig:meme_phase_transition_indi}
\end{figure}

To find the probability of at least 25\% of the populations share a meme. I run the model with B(0) = 1:1:999 for 100 times to final time = 1000. and plot the probability in a heat map. 

\begin{figure}[H] %result plot of individual
\centering
\includegraphics[width = 16 cm, height = 13cm]{probability250_100sim.png}
\caption{probability of more than 250 people are sharing with different B0 and time from 0 to 1000}
\label{fig:probability250}
\end{figure}

\newpage


%%%%%%%2.2 with different rules on bored 

\subsection{changed the condition of a bored person}
One condition was added to the bored person. A bored person will pick on person a random from the population with probability of q. If that person is resting then the bored person will become resting. otherwise she will continue to be bored. It was simulated in matlab for 1000 times with a population of 1000 to time = 2000 and at the same time plot together with the mean field model. The results were very different compared to the previous simulation. It seems that the model shows oscillation of bored person and sharing person over time but not in the simulation. In the model, we allow decimals for the number of persons. that's why the number of bored persons can slowly increase. In the simulation, it started with 1 bored person, and if that person meets someone at rest with probability of q. Then this bored person will be resting too. With this condition, a bored person soon finds a person at rest and then bored persons will be 0. The sharing person will increase gradually as a resting person will find a meme with p = 0.001. In the simulation one person can only be in one of the three states, that's why the simulations shows different results. 

\begin{figure}[H] %rmeme2 1000 runs
\centering
\includegraphics[width = 16 cm, height = 13cm]{memes2_sim_1000times.png}
\caption{100 simulations of memes with different rules for bored person over t = 2000  }
\label{fig:meme2sim1000}
\end{figure}

\begin{figure}[H] %meme2 1000 runs with model
\centering
\includegraphics[width = 16 cm, height = 13cm]{memes2_withmodel1000.png}
\caption{100 simulations of memes with different rules for bored person over t = 2000  with mean field model plot}
\label{fig:meme2sim1000model}
\end{figure}

A phase transtion over q = 0.01:0.01:1 was made for the simulation. it is interesting to see that with probability q a person interact with another either to share or transtion from bored to rest. from q = 0.01 to q = 0.1, the total number of sharing persons at t = 1000 increase rapidly and almost everyone was sharing in the end when p $>$ 0.1. 
\begin{figure}[H] %meme2 phase
\centering
\includegraphics[width = 16 cm, height = 13cm]{phasetransition2.png}
\caption{phase transition of total sharing person with simulation with t = 1000 and different q condition }
\label{fig:phasetransition2}
\end{figure}

\begin{figure}[H] %meme2 phase indi
\centering
\includegraphics[width = 16 cm, height = 13cm]{indi_phase2t1000withq.png}
\caption{plot of sharing person in individual run at t = 1000 }
\label{fig:phasetransition2indi1000}
\end{figure}

\subsection{simulate on grids of 40x40 for spread of memes}
In this part, we use the model above and in 2.2 to simulate that the person can only interact with the neighbours. the boundary conditions are set to be periodic. so that it goes from left to right, right to left, up to bottom, and bottom to up. After several runs. it can be observed that the number of sharer are increasing steadily. while the number of bored people tend to remain around at 1.the only way that bored person increase is that a sharing person meets a bored person. in theory the probability is p = 1* 0.01*(1/1600). For the majority resting population, with a chance of 0.001 to discover a memes and become a sharer. in the initial state, there were 1588 resting people, that means that in theory 1.5 people will become a sharer. That's why it grows very fast, when t gets to 1000. almost everyone is sharing. 

\begin{figure}[H] %meme2 phase indi
\centering
\includegraphics[width = 16 cm, height = 13cm]{memegrid.png}
\caption{initial condition for simulation}
\label{fig:memegrid0}
\end{figure}

\begin{figure}[H] %meme2 phase indi
\centering
\includegraphics[width = 16 cm, height = 13cm]{memegrid100.png}
\caption{simulation of spread of meme to t = 100}
\label{fig:memegrid100}
\end{figure}

The video of the simulation can be found at 
\url{<https://www.youtube.com/watch?v=WMfI2P52ros>}

%%%%%%%%%%%%%%%%%%
%%%%%%%%%%%%%%%%%
%%%%%%%%%%%%%%%%%appendix
\newpage
\section{Appendix}

\subsection{1firing brain code in matlab}
\singlespacing
\begin{itemize}
\item {\large simulate single time of fire brain}
\lstinputlisting{firebrain1.m}
\vspace{1cm}

\item {\large transition function}
\lstinputlisting{transit.m}
\vspace{1cm}

\item {\large initial state}
\lstinputlisting{random_start.m}
\vspace{1cm}

\item {\large simulate 100 times of firing brain}
\lstinputlisting{firebrain1_simulate100.m}
\vspace{1cm}

\item {\large cell that move forward at one cell per time preserving the same shape}
\lstinputlisting{task2_1.m}
\vspace{1cm}

\item {\large cell that move forwad at one cell per time, launching other shapes behind them}
\lstinputlisting{task2_2a.m}
\vspace{1cm}

\item {\large move forward at a rate of less than one cell per time step}
\lstinputlisting{task2_3.m}
\vspace{1cm}

\item {\large oscillate shape}
\lstinputlisting{task2_4.m}
\vspace{1cm}


\item {\large my cellular automata}
\lstinputlisting{reproduceown.m}
\vspace{1cm}

\item {\large my cellular automata transit}
\lstinputlisting{transitown.m}
\vspace{1cm}


\end{itemize}



\subsection{Spread of memes}

\singlespacing
\begin{itemize}

\item{\large simulation of spread of memes}
\lstinputlisting{sim_meme1.m}
\vspace{1cm}

\item{\large run single spread of memes }
\lstinputlisting{runmeme.m}
\vspace{1cm}

\item{\large mean field model}
\lstinputlisting{sim_meme_model.m}
\vspace{1cm}

\item{\large phase transition}
\lstinputlisting{sim_meme_phase.m}
\vspace{1cm}


\item{\large probability for at least 25\% are sharing}
\lstinputlisting{plargethan250.m}
\vspace{1cm}


\item{\large simulation of spread of memes with new rules}
\lstinputlisting{sim_meme2.m}
\vspace{1cm}

\item{\large run single spread of memes }
\lstinputlisting{runmeme_newbored.m}
\vspace{1cm}

\item{\large mean field model}
\lstinputlisting{sim_meme_model2.m}
\vspace{1cm}

\item{\large phase transition for new rules}
\lstinputlisting{sim_meme_phase2.m}
\vspace{1cm}

\item{\large lattice simulation for memes}
\lstinputlisting{memegrid.m}
\vspace{1cm}

\item{\large transition function for memes}
\lstinputlisting{transit_meme.m}
\vspace{1cm}


\end{itemize}


\end{document}