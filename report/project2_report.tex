\documentclass[12pt]{article}
\usepackage[swedish,english]{babel}
\usepackage[utf8x]{inputenc}
\usepackage{amsmath}
\usepackage{graphicx}
\usepackage{float} %for insert graph
\usepackage{lscape}
\usepackage{rotating}
\usepackage[colorinlistoftodos]{todonotes}
%\usepackage[margin=1in]{geometry} %set page margin
\usepackage[bottom=1.25in, top=1.25in]{geometry}
%\addtolength{\topmargin}{0.25in}
%\addtolength{\bottommargin}{0.25in}
\usepackage{hyperref} %insert to link to email address
\usepackage{setspace}
\setlength\parindent{24pt} %set indentation
\usepackage{amssymb} %for maths symbols
\usepackage{cases} %for numbering in cases
  %define matlab style
\usepackage{listings}
\usepackage{color} %red, green, blue, yellow, cyan, magenta, black, white
\definecolor{mygreen}{RGB}{28,172,0} % color values Red, Green, Blue
\definecolor{mylilas}{RGB}{170,55,241}
\usepackage{grffile} %to avoid showing the file names of figures

\lstset{language=Matlab,%
    %basicstyle=\color{red},
    breaklines=true,%
    morekeywords={matlab2tikz},
    keywordstyle=\color{blue},%
    morekeywords=[2]{1}, keywordstyle=[2]{\color{black}},
    identifierstyle=\color{black},%
    stringstyle=\color{mylilas},
    commentstyle=\color{mygreen},%
    showstringspaces=false,%without this there will be a symbol in the places where there is a space
    numbers=left,%
    numberstyle={\tiny \color{black}}% size of the numbers
    %numbersep=9pt, % this defines how far the numbers are from the text
    %emph=[1]{for,end,break},emphstyle=[1]\color{red}, %some words to emphasise
    %emph=[2]{word1,word2}, emphstyle=[2]{style},    
}


\begin{document}

%%%%%%%%%%%
%%TITLE PAGE%%%
%%%%%%%%%%%%
\begin{titlepage}

\newcommand{\HRule}{\rule{\linewidth}{0.5mm}} % Defines a new command for the horizontal lines, change thickness here

\center % Center everything on the page
 
%----------------------------------------------------------------------------------------
%   HEADING SECTIONS
%----------------------------------------------------------------------------------------
%Logo First
%----------------------------------------------------------------------------------------
%   LOGO SECTION
%----------------------------------------------------------------------------------------

\includegraphics{UU_logo.eps}\\[0.5cm] % Include a department/university logo - this will require the graphicx package
 
\textsc{\huge Modelling complex systems}\\[0.5cm] % Major heading such as course name


%----------------------------------------------------------------------------------------
%   TITLE SECTION
%----------------------------------------------------------------------------------------

\HRule \\[0.4cm]
{\huge \bfseries Project 2}\\[0.2cm] % Title of your document
{\Large Population Dynamics \par Groups of Friends \par Network Epidemics \par Flocks and Predators }
\\

\HRule \\[0.4cm]
%{\Large Population Dynamics \par Groups of Friends \par Network Epidemics \par Flocks and Predators }

 
%----------------------------------------------------------------------------------------
%   AUTHOR SECTION
%----------------------------------------------------------------------------------------

{\huge Peili Guo\\} %insert your name 
{\large \href{mailto:Peili.Guo.7645@student.uu.se}{Peili.Guo.7645@student.uu.se}}
\\[2cm] %insert page break length

%----------------------------------------------------------------------------------------
%   DATE SECTION
%----------------------------------------------------------------------------------------

{\Large \today}\\[2cm] % Date, change the \today to a set date if you want to be precise


%----------------------------------------------------------------------------------------

\vfill % Fill the rest of the page with whitespace

\end{titlepage}

\newpage
%%%%%%%%%%%%
%%%start report 
%%%1. population dynamics
%%%%%%%%%%%

\section{Population Dynamics}
\doublespacing
In this part, we model the population with a stochastic model. There are n resource sites in the model world, and at time t =0, the population is $A_{0}$ and they are assigned randomly to one resource site. At each t step, the population rules are, if there are exactly two individuals on the same site. They reproduce b offsprings and these offsprings are assigned randomly to resources sites. If the number of individuals on a resources sites is any number other than 2, no offspring will be reproduced. 

\subsection{Matlab model}

To begin, we can run this with different parameters and simulate in Matlab and observe the total number of population at different time step. The different parameters are: 

\begin{enumerate}
\item b: number of offspring if reproduce
\item n: total of number of resource sites
\item $A_{0}$: initial population
\item t: the time steps we want to simulate the model
\end{enumerate}


When set the initial population to a small number relative to the resource site, it would be difficult to have 2 individuals at the same site for reproduce, and even if they reproduce a large number of offspring the population dies out very quick. 
below are some plots showing the total number of population at different time steps with different parameters. 

\begin{figure}[H] %figure 1 at x0=10 b= 50
\centering
\includegraphics[width = 12 cm, height = 9cm]{single_run n1000 x010 b50.png}
\caption{initial population of 10 b = 50 not reproducing}
\label{fig:p1s1}
\end{figure}

\begin{figure}[H] %figure 2 x0 = 50 b =30 
\centering
\includegraphics[width = 12 cm, height = 9cm]{single_run n1000 x050 b30.png}
\caption{initial population of 50 and b =30 not reproducing}
\label{fig:p1s2}
\end{figure}

\begin{figure}[H] %figure3 x0 = 50 b= 50
\centering
\includegraphics[width = 12 cm, height = 9cm]{single_run n1000 x050 b50.png}
\caption{initial population of 50 and b = 50, in the beginning reproduce but quickly dies}
\label{fig:p1s3}
\end{figure}

\begin{figure}[H] %figure4 x0 = 100 b = 20
\centering
\includegraphics[width = 12 cm, height = 9cm]{single_run n1000 x0100 b20.png}
\caption{initial population of 100 and b = 20, it dies}
\label{fig:p1s4}
\end{figure}

\begin{figure}[H] %figure5 x0 = 100 b = 50
\centering
\includegraphics[width = 12 cm, height = 9cm]{single_run n1000 x0100 b50.png}
\caption{initial population of 100 and b = 50, it reproduce in the beginning but quickly dies}
\label{fig:p1s5}
\end{figure}

\begin{figure}[H] %figure 6 x0 1000 b = 10
\centering
\includegraphics[width = 12 cm, height = 9cm]{single_run n1000 x01000 b10.png}
\caption{initial population of 1000 and b = 10, it show that the population oscillates around 2500}
\label{fig:p1s6}
\end{figure}

\begin{figure}[H] %figure 7 x0 1000 b =20
\centering
\includegraphics[width = 12 cm, height = 9cm]{single_run n1000 x01000 b20.png}
\caption{initial population of 1000 and b = 20, the populations starts to get chaotic}
\label{fig:p1s7}
\end{figure}

\begin{figure}[H] %figure 8 x0 1000 b =50
\centering
\includegraphics[width = 12 cm, height = 9cm]{single_run n1000 x01000 b50.png}
\caption{initial population of 1000 and b = 500, the populations dies very quick}
\label{fig:p1s8}
\end{figure}


To further study this model, we run it with $A_{0} = 1000$, n= 1000, b = 1, 2, 3, 4, 5,...,48, 49, 50. and at each b value, run the simulation 100 times and plot a phase transition diagram. and we can see that when b = 5-15, the population increases steady and oscillates around a number. When $15<b<35$, the population will get more chaotic and at one time a lot of sites are reproducing, and then the next time steps, they dies because of overcrowding. When $b>35$ the population will dies very quick due to overcrowding. 


\begin{figure}[H] %figure 9 phase transition diagram
\centering
\includegraphics[width = 16 cm, height = 13cm]{q1_phase_r100.png}
\caption{initial population of 1000 and b = 500, the populations dies very quick}
\label{fig:p1q1pd}
\end{figure}

\newpage
\subsection{Mean field model}
We assume the sites are independent and number of individuals at sites is poisson distributed. 
I plot selected mean-field model here for b = 5, 10, 15, 20, 35, 50. The model are after the plot

\begin{figure}[H] %figure 10 mean field model
\centering
\includegraphics[width = 18 cm, height = 14cm]{meanfield_model.png}
\caption{mean field model with selected b value}
\label{fig:p1q1mf}
\end{figure}

\begin{numcases}{}
	E(A_{t+1}|A_{t}) = n \times \sum_{k=0}^{n} P_{k} \times \phi_{k} \\
	\phi_{k} = \begin{cases}
		b, if k =2\\
		0, if k \neq 2
	\end{cases}
\end{numcases}



following the mean field model, and the population only reproduce if there are exactly 2 individuals at same site, we can write the following:

\begin{equation}
	A_{t+1} = n \times P(2 at site, A_{t}) \times b\\
\end{equation}

\begin{equation}
	A_{t+1} = n \times \frac{(\frac{A_{t}}{n})^{2} \times e^{-\frac{A_{t}}{n}}} {2} \times b
\end{equation}

For steady state, we have $A_{t+1} = A_{t}$.\\

\begin{equation}
	A_{t} = n \times \frac{(\frac{A_{t}}{n})^{2} \times e^{-\frac{A_{t}}{n}}} {2} \times b
\end{equation}

When $A_{t} = 0$, left hand side always equal to right hand side, they are both 0. When $A_{t}  \neq 0$, we have

\begin{equation} \label{eq1}
	1 = \frac{b \times A_{t}}{2n} \times e^{-\frac{A_{t}}{n}}
\end{equation}

Let's rearrange equation (6) and let $\frac{A_{t}}{n} = x $, we have:

\begin{equation}
	1 = \frac{b}{2} \times x \times e^{-x}
\end{equation}

\begin{equation}
	x \times e^{-x} = \frac{b}{2}
\end{equation}

To find the conditions in terms of b for the existence of two further non-zero steady states, we need to find that $f1(x) = x \times e^{-x}$  has intersection with a horizontal line. b need to fulfill the condition that $\frac{2}{b} < 0.3679$, therefore we get: $b > 5.4366$, as b needs to be integer, we get $b > 5$. 


\begin{figure}[H] %figure 11 plot of function
\centering
\includegraphics[width = 16 cm, height = 13cm]{q1_2plotfunction.png}
\caption{plot of f1(x) = x * exp(-x)}
\label{fig:p1q1plotfun}
\end{figure}


\newpage
\subsection{Lyapunov exponent}

From the previous section, we have derived the equation for $A_{t}$. that we have 
\begin{equation}
	f(A_t) = n \times \frac{(\frac{A_{t}}{n})^{2} \times e^{-\frac{A_{t}}{n}}} {2} \times b
\end{equation}

\begin{equation}
	f(a) = \frac{b}{2n} \times a^{2} \times e^{-\frac{a}{n}}
\end{equation}

\begin{equation}
	f'(a) = \frac{b}{2n} \times (2a \times e^{-\frac{a}{n}} + a^{2} \times e^{-\frac{a}{n}} \times(- \frac{1}{n}))
\end{equation}

\begin{equation}
	f'(a) = \frac{b}{2n} \times (2a \times e^{-\frac{a}{n}} - \frac{a^{2}}{n} \times e^{-\frac{a}{n}} )
\end{equation}

\begin{equation}
	|\Delta a_{n}| = |\Delta a_{0}| e^{\lambda n}
\end{equation}

to compute Lyapunov exponent $\lambda$ numerically, we use the following:
\begin{equation}
	\lambda = \frac{1}{n} \sum_{t=0}^{n-1} ln |f'(a_{t})|
\end{equation}

The Lyapunov exponent are plotted below: we can see that when $\lambda > 0$ the diverges, that's when b is between 21-30, the population become chaotic. When $b \leq 5$ and $b \geq 31$ $\lambda$ is -infinity that means the system converge very fast, and this explains why the population become extinct very fast for large b. 

\begin{figure}[H] %figure 12 lyaponov exponent
\centering
\includegraphics[width = 16 cm, height = 13cm]{lyapunov_exp.png}
\caption{plot of Lyapunov exponent}
\label{fig:p1q1lya}
\end{figure}


%%%%%%%%%%%%%%%%%%%%%%%%%%part 2
%%%%%%%groups of friends

\newpage
\section{Groups of friends}
\doublespacing
In this part, we used a model to simulate how students make friends. Initially, there are N students, and they are all alone in a group contains only one student. At each time step, a group is picked:

\begin{enumerate}
\item If the picked group contains only one student, this students will join a group with size k with probability of k/N.  
\item if the picked group size i ($i > 1$), then this group will split up with probability of ri. 
\end{enumerate}


\subsection{simulations in Matlab}
First, we simulate the above model with 100 students, r = 0.01 and 100 replications. First, we plot the group size vs frequence on a log log scale. Second, we get the relative frequency with group size histogram taken on a log scale that is we take groups size [1,2,4,8,16,32,64,128] and plot the group size vs relative frequency on a log log plot. in figure \ref{fig:loglog1}, we can observe that the group distribution following the power law. 


\begin{figure}[H] %f13 loglog with groupsize 1:1:100
\centering
\includegraphics[width = 12 cm, height = 9cm]{loglog2.png}
\caption{log log plot of group size vs frequency}
\label{fig:loglog2}
\end{figure}

\begin{figure}[H] %f14 loglog with groupsize 1 2 4 8 16 32
\centering
\includegraphics[width = 12 cm, height = 9cm]{loglog1.png}
\caption{log log plot of group size vs relative frequency}
\label{fig:loglog1}
\end{figure}

To investigate the how the group size distribution changes with r, we run the model with r from 0.001:0.001:0.1 with 100 replications. In \ref{fig:loglog3}, I choose some r values to plot shows how the group size changes, and in \ref{fig:loglog4}, the whole group distribution is shown on a 3d plot, it might be a little difficult to from the colourful map, but we can observe the z-axis value that the relative frequency of the group. \par
When r is small at 0.001 we have groups of size 1 and 2 around $\frac{1}{4}$ of the population and some large groups. when we increase the r, then the number of groups of 1 become the dominant group in the distribution and large group appear less and less. With a bigger r, the larger group will split up if it was chosen at time t. For example, setting r = 0.1, it means that groups of size 10 + will slip up at the probability of 1 if chosen at time t. 

\begin{figure}[H] %f15 loglog with r value
\centering
\includegraphics[width = 14 cm, height = 11cm]{logloggr100.png}
\caption{log log plot of group size vs relative frequency with different r}
\label{fig:loglog3}
\end{figure}

\begin{figure}[H] %f15 loglog with r value
\centering
\includegraphics[width = 16 cm, height = 13cm]{hmq2.png}
\caption{the distribution of groups with different r showing on a 3d plot}
\label{fig:loglog4}
\end{figure}





%%%%%%%2.2 master equation
\newpage
\subsection{Master Equation}



%%%%%%%%%%%%
%%%%%%%%%%%%ATTENTION NEEDS HERE IN THIS PART
%%%%%%%%%%%%%%%



%%%%%%%%%%%%%%%%%%%%%%%%%%part 3
%%%%%%%Network Epidemics

\newpage
\section{Network Epidemics}
\doublespacing

\subsection{random undirected social network}

An undIrected networking was made with 5000 students and a link density of 0.0016. we use matrix A to represent the network, as the network is undirected, we have A as a symmetric matrix, and with a link density of 0.0016, we generate A and plot the histogram of the degree distribution. The average degree of this network is around 8 and the degree follows a binomial distribution. 

\begin{figure}[H] %f16 frequency of histogram
\centering
\includegraphics[width = 12 cm, height = 9cm]{randomf.png}
\caption{Frequency of the connection degree}
\label{fig:randomconnect}
\end{figure}

\newpage
\subsection{random undirected social network}
















%%%%%%%%%%%%%%%%%%
%%%%%%%%%%%%%%%%%
%%%%%%%%%%%%%%%%%appendix
\newpage
\section{Appendix}

\subsection{1firing brain code in matlab}
\singlespacing
\begin{itemize}
\item {\large simulate single time of fire brain}
\lstinputlisting{firebrain1.m}
\vspace{1cm}

\item {\large transition function}
\lstinputlisting{transit.m}
\vspace{1cm}

\item {\large initial state}
\lstinputlisting{random_start.m}
\vspace{1cm}

\item {\large simulate 100 times of firing brain}
\lstinputlisting{firebrain1_simulate100.m}
\vspace{1cm}

\item {\large cell that move forward at one cell per time preserving the same shape}
\lstinputlisting{task2_1.m}
\vspace{1cm}

\item {\large cell that move forwad at one cell per time, launching other shapes behind them}
\lstinputlisting{task2_2a.m}
\vspace{1cm}

\item {\large move forward at a rate of less than one cell per time step}
\lstinputlisting{task2_3.m}
\vspace{1cm}

\item {\large oscillate shape}
\lstinputlisting{task2_4.m}
\vspace{1cm}


\item {\large my cellular automata}
\lstinputlisting{reproduceown.m}
\vspace{1cm}

\item {\large my cellular automata transit}
\lstinputlisting{transitown.m}
\vspace{1cm}


\end{itemize}



\subsection{Spread of memes}

\singlespacing
\begin{itemize}

\item{\large simulation of spread of memes}
\lstinputlisting{sim_meme1.m}
\vspace{1cm}

\item{\large run single spread of memes }
\lstinputlisting{runmeme.m}
\vspace{1cm}

\item{\large mean field model}
\lstinputlisting{sim_meme_model.m}
\vspace{1cm}

\item{\large phase transition}
\lstinputlisting{sim_meme_phase.m}
\vspace{1cm}


\item{\large probability for at least 25\% are sharing}
\lstinputlisting{plargethan250.m}
\vspace{1cm}


\item{\large simulation of spread of memes with new rules}
\lstinputlisting{sim_meme2.m}
\vspace{1cm}

\item{\large run single spread of memes }
\lstinputlisting{runmeme_newbored.m}
\vspace{1cm}

\item{\large mean field model}
\lstinputlisting{sim_meme_model2.m}
\vspace{1cm}

\item{\large phase transition for new rules}
\lstinputlisting{sim_meme_phase2.m}
\vspace{1cm}

\item{\large lattice simulation for memes}
\lstinputlisting{memegrid.m}
\vspace{1cm}

\item{\large transition function for memes}
\lstinputlisting{transit_meme.m}
\vspace{1cm}


\end{itemize}


\end{document}